\documentclass[12pt,a4paper]{article}
\usepackage[T1]{fontenc}
\usepackage[utf8]{inputenc}
\usepackage[italian]{babel}
\usepackage{lmodern}
\usepackage{graphicx}
\usepackage{minted}
\usepackage{hyperref}

\setlength{\parindent}{2em}
\setlength{\parskip}{1em}

\setminted[python]{frame=lines,framesep=1em}

\newcommand{\mrcnn}{Mask-R\_CNN}

\begin{document}

\title{Riconoscimento Documenti}
\author{Simone Cimarelli \and Vittorio Mignini \and Luca Moroni}
\date{9 luglio 2020}

\maketitle

\begin{abstract}
    Si descrive l'implementazione di una libreria basata su \mrcnn\ e
    tesseract per il riconoscimento interattivo di documenti
    identificativi e tessere.
    --> modificare %TODO
\end{abstract}

\section{Obbiettivi}

Implementare un applicativo che tramite l'utilizzo di un dispositivo di
input video sia in grado di riconoscere e di identificare un insieme di
documenti a priori definiti e di riconoscere delle porzioni di testo
presenti nel documento precedentemente identificato.

\section{Analisi dei Requisiti}
\subsection{Analisi Requisiti Utente}

\begin{itemize}
    \item Ottenere un elenco dei documenti visibili all'interno di
        un'immagine e le componenti testuali specifiche in esso
        riconoscibili.
    \item Avere la possibilità di definire la natura dei documenti
        oggetto del riconoscimento e delle informazioni testuali
        rilevanti in essi contenute.
    \item Mettere a disposizione un interfaccia grafica semplice e
        minimale per l'acquisizione di video e immagini
\end{itemize}

\subsection{Analisi Requisiti Sistema}

\begin{itemize}
    \item Implementare delle tecnologie per l'object detection e ocr
        allo stato dell'arte per erogare in modo efficiente il servizio
        richiesto dal committente.
    \item Fornire strumenti che mettono a disposizione la possibilita di
        rendere scalabile il numero e la tipologia di documenti
        riconoscibili dall'applicativo
    \item L'applicativo finale \textit{dovrebbe} essere portabile e non
        strettamente dipendente da una particolare architettura
\end{itemize}

\subsection{Glossario e Dominio applicativo}

Il prodotto risponde all'esigenza di individuare la tipologia ed
eseguire il parsing dei prinicipali dati testuali esposti in vari
documenti di riconoscimento, tramite webcam.
I domini applicativi di tale sistema e le problematiche ad essi connesse
sono sono quelli dell'ambito della computer vision e dell'OCR.
L'utente finale avrà a disposizione un'interfaccia grafica attraverso la
quale, in modo intuitivo ed efficiente, potrà avere accesso al servizio
erogato.

\begin{description}
    \item[Computer Vision] Insieme di metodologie algoritmiche e
        matematiche che hanno come obiettivo quello di acquisire delle
        conoscenze ad alto livello partendo da immagini o video. Nel
        nostro caso capire se e dove nelle immagini passate
        all'applicativo è presente un documento del tipo prestabilito.
    \item[Object Detection] Tecnologia relativa all'ambito della
        computer vision, che propone una soluzione algoritmica per il
        riconoscimento di istanze riguardanti oggetti di classi
        predefinite in immagini digitali e video.
    \item[OCR] Optical Character Recognition, metodologie software per
        il riconoscimento di caratteri contenuti in un documento e
        trascrizione di tali simboli in caratteri equivalenti.
\end{description}

\subsection{Modellazione Concettuale del Sistema}

Il compito di interagire con l'utente finale sarà a carico di una
componente di interfaccia grafica ``frontend'' che si occuperà
dell'accesso alla videocamera e mostrerà uno storico dei documenti
riconosciuti e in corso di riconoscimento in tempo reale, mentre
delegherà il compito di applicare e interpretare i risultati delle
primitive di riconoscimento e OCR a una componente ``buisiness logic''
configurabile dal committente potendo scegliere il tipo di documenti da
riconoscere e le carateristiche testuali da ricercare.

\section{Architettura del Sistema}

\subsection{Scelte architetturali}

Abbiamo optato per lo sviluppo di una soluzione web, Il nostro
applicativo si suddivide in tre macro parti:

\begin{itemize}
    \item la parte frontend, sviluppata tramite tecnologie browser
        compatibili, js html CSS
    \item la parte backend, sviluppata in python tramite la libreria
        flask per l'erogazione di servizi web
    \item la parte di core bussness, ovvero il core dell'applicativo,
        sviluppata in python per l'utilizzo di ocr e object
        detection
\end{itemize}

Nella progettazione architetturale di tale applicativo, si è optato per
l'utilizzo di librerie testate e rappresentanti lo stato dell'arte,
inerentemente all'ambito applicativo sopracitato.\\
Per quanto riguarda l'object detection, abbiamo selezionato la libreria
\textbf{\mrcnn} una libreria che oltre a effettuare object detection mette a
disposizione anche segmentazione di immagini, una feature aggiuntiva che
oltre a ritornare un rettangolo contenente un oggetto di una determinata
classe nell'immagine ne ritorna anche la regione di pixel che lo
rappresenta, tale feature comunque non è utilizzata nel nostro sistema,
ma è tornata utile in fase di training in quanto possiamo allenare la
nostra rete direttamente tramite i segmenti di documenti, evitando la
presenza dei polpastrelli che in alcuni casi possono portare la nostra
rete fuori rotta.\\
Invece per effettuare l'ocr abbiamo optato per la libreria tesseract,
resa disponibile in python tramite una libreria wrapper chiamata
pytesseract.

Le librerie appena citate hanno portato alla necessità di importarne
altre tra cui, tensorflow$=$1.x, flask$=$2.0.8 sulle quali si basa
\mrcnn, opencv per la gestione e la codifica di immagini.

Per la gestione delle dipendenze in fase di sviluppo abbiamo utilizzato
due strumenti propri dell'ambiente python, pipenv che mette a
disposizione un ambiente virtuale contenente determinate librerie in
modo asettico rispetto a quelle di sistema e che rende ripetibile su
altre macchine la configurazione da noi utilizzata con semplici
passaggi. pyenv che invece mette a disposizione delle versioni di python
proprie in modo da non andare a modificare quella di sistema nel caso
sia necessaria un versione non corrispondente con quella di sistema, ad
esempio tensorflow$=$1.x è utilizzabile solamente con versioni di
python$\leq$3.7.x con pyenv tale requisito puo essere soddisfatto su
qualsiasi macchina.

\subsection{Mask}

There are two stages of \mrcnn. First, it generates proposals about
the regions where there might be an object based on the input image.
Second, it predicts the class of the object, refines the bounding box
and generates a mask in pixel level of the object based on the first
stage proposal. Both stages are connected to the backbone structure.

A light weight neural network called RPN scans all FPN top-bottom
pathway( hereinafter referred to feature map) and proposes regions which
may contain objects.

At the second stage, another neural network takes proposed regions by
the first stage and assign them to several specific areas of a feature
map level, scans these areas, and generates objects
classes(multi-categorical classified), bounding boxes and masks.

--> aggiungere descrizione piu dettagliata di mask-r\_cnn e foto descrittiva

\subsection{Training}

Essendo il nostro corebusiness basato su una rete neurale per la
computer vision implementata tramite la libreria \mrcnn, una rete
neurale molto complessa sulla quale facciamo transfer learning, ovvero
alleniamo solamente gli strati piu estrerni per adattare i pesi e la
struttura della stessa al riconoscimento di un certo numero di classi
definibili da noi sviluppatori e in ultima istanza dall'utente
utilizzatore, ci siamo appoggiati al servizio colab di google per il
training della rete tramite la computazione limitata ma estremamente
potente messa a dispozione da tale servizio.\\
Per rendere ripetibile tutto ciò abbiamo reso disponibile il dataset da
noi utilizzato e il file jupiter (tecnologia su cui si basa colab), cosi
da poter ripetere la fase di training facendo l'opportuno tuning sui
parametri e cosi da poter anche aggiungere nuove classi di oggetti e
nuove istanze per il training. Infine si rimanda al link
dell'applicativo web necessario alla creazione di annotazioni per le
immagini utilizzate come istanze di training.
\url{http://www.robots.ox.ac.uk/~vgg/software/via/}

--> struttura di dataset con esempio di file di via\_regions.json

si faccia riferimento al notebook jupiter a fine report: --> nota % TODO

\subsection{Frontend}

--> descrizione frontend

\subsection{Backend}

--> descrizione backend

\subsection{Core Business}

Il parte di core business è implementata in un modulo Python. Le
funzionalità di riconoscimento sono rese disponibili tramite istanza di
una classe \texttt{Detectron}, inizializzata tramite una struttura di
configurazione definibile dall'utente in congiunzione con i pesi della
rete neurale, per la massima estensibilità.

Una volta inizializzato, un oggetto \texttt{Detectron} offre un mketodo
\texttt{recognize} che implementa la funzione di riconoscimento.\\
\texttt{recognize} accetta un oggetto \texttt{bytes}-like che
rappresenta l'immagine in un qualsiasi formato supportato dalla libreria
OpenCV2. Dopo aver decodificato l'immagine questa viene passata
attraverso la rete neurale \mrcnn\ precedentemente inizializzata, che
individua delle ROI associate ai tipi di documento configurati, a ognuna
delle quali sarà associato un dizionario. L'area di ogni ROI viene
estratta dall'immagine e memorizzata alla chiave ``\texttt{snapshot}'',
codificata in PNG in un oggetto \texttt{bytes}.

Su questi snapshot viene poi applicato l'algoritmo di OCR della libreria
tesseract. Quest'ultimo a sua volta restituirà una serie di record che
rappresentano il testo trovato, il grado confidenza e la relativa
posizione all'interno della ROI. Questi record sono identificati in base
alla posizione o al formato, specificato tramite regex. Il testo dei
record così individuati è aggiunto a un dizionario sotto il nome di
``\texttt{attributes}''. L'oggetto di configurazione definisce quali di
questi ``\texttt{attributes}'' sono necessari al riconoscimento del
documento e solo quando tutti questi sono effettivamente stati
individuati il documento viene marcato come valido. Uno degli attributi
necessari può essere designato come chiave primaria, quando un documento
è valido il valore della chiave primaria viene restituito nel dizionario
sotto il nome di ``\texttt{primaryKey}'' e può essere utilizzato per
l'identificazione univoca del documento appena scansionato. Si consiglia
di indicare come chiave primaria il codice o numero identificativo
univoco del documento, quando presente.\\
La funzione \texttt{recognize} ritorna in output una lista di questi
dizionari.

\pagebreak

\subsubsection{Esempio di configurazione}
\inputminted{python}{config.py}

\begin{figure}[p]
  \caption{Diagramma delle Classi}
  \centering
  \includegraphics[width=\textwidth,height=\textheight,keepaspectratio]{uml_class.pdf}
\end{figure}

--> aggiungere descrizione diagramma di classe

\begin{figure}[p]
  \caption{Diagramma di Stato}
  \centering
  \includegraphics[width=\textwidth,height=\textheight,keepaspectratio]{uml_state.pdf}
\end{figure}

--> aggiungere descrizione diagramma di stato per oggetto Detectron

\begin{figure}[p]
  \caption{Diagramma di Caso d'Uso}
  \centering
  \includegraphics[width=\textwidth,height=\textheight,keepaspectratio]{uml_use_case.pdf}
\end{figure}

--> aggiungere descrizione diagramma caso d'uso

\begin{figure}[p]
  \caption{Diagramma di Sequenza}
  \centering
  \includegraphics[width=\textwidth,height=\textheight,keepaspectratio]{uml_seq.pdf}
\end{figure}

--> aggiungere descrizione diagramma di sequenza (aggiungendo classe di libreria mrcnn)

\begin{figure}[p]
  \caption{Diagramma di Comunicazione}
  \centering
  \includegraphics[width=\textwidth,height=\textheight,keepaspectratio]{uml_comm.pdf}
\end{figure}

--> aggiungere descrizione diagramma di comunicazione (aggiungendo classe di libreria mrcnn)

\begin{figure}[p]
  \caption{Diagramma di Attività}
  \centering
  \includegraphics[width=\textwidth,height=\textheight,keepaspectratio]{uml_activity.pdf}
\end{figure}

--> aggiungere descrizione diagramma di attivita

\pagebreak

\section{Design Pattern}

Un design pattern che in potenza poterebbe essere applicato è il
Protection Proxy. Inizialmente non eravamo convinti della thread safness
della funzione \texttt{recognize} nella classe \texttt{Detectron}. Una
possibile soluzione consisterebbe nel creare una classe
\texttt{DetectronPool}, la quale nell'inizializzazione carica una pool
di oggetti \texttt{Detectron}. Tale classe avrà un metodo
\texttt{recognize} che rimanderà all'omonimo metodo di
\texttt{Detectron}, il quale gestirebbe l'allocazione della risorsa al
thread chiamante in modo trasparente tramite un semaforo contatore
inizializzato al numero di \texttt{Detectron} nella pool, garantendo la
mutua esclusione nell'esecuzione del metodo \texttt{recognize} in
maniera totalmente trasparente per l'utilizzatore.

\begin{figure}[h]
  \caption{Diagramma di Classe Proxy}
  \centering
  \includegraphics[width=\textwidth,height=\textheight,keepaspectratio]{uml_proxy.pdf}
\end{figure}

\pagebreak

\section{Training Notebook} % TODO spostare dove appropriato
\setminted[python]{linenos,breaklines,breakbefore=/,frame=single,framesep=2mm}

Segue una copia commentata di un notebook jupyter che è stato utilizzato
per eseguire il training del dataset di esempio in Google Colab. Per
l'ultima versione dell'originale consultare la repo git.

\subsection{Preparazione dei prerequisiti}

Configuriamo la versione per le librerie utilizzate che ne richiedono
una in particolare.

\begin{minted}{python}
print("DOCUMENT RECOGNITION Training")
%tensorflow_version 1.x
!pip install keras==2.0.8
\end{minted}

Cloniamo la repo git di \mrcnn\ e la installiamo, dopodiché proseguiamo
il lavoro nella directory della repo stessa

\begin{minted}{python}
# setup the library and its dependencies
!git clone https://github.com/matterport/Mask_RCNN.git
%cd Mask_RCNN
!python setup.py install
!pip install -r requirements.txt
\end{minted}

Montiamo la cartella Google Drive nella quale il dataset è stato
preventivamente caricato.

\begin{minted}{python}
from google.colab import drive
drive.mount('/content/drive')
\end{minted}

Scarichiamo un file di pesi preallenato su Microsoft COCO, un dataset
pensato per il riconoscimento di oggetti in contesti non isolati sul
quale faremo transfer learning per le nostre classi di documenti.

\begin{minted}{python}
!wget https://github.com/matterport/Mask_RCNN/releases/download/v2.0/mask_rcnn_coco.h5
\end{minted}

\subsection{Inizializzazione}

Importiamo le librerie e prepariamo i percorsi.

\begin{minted}{python}
import os
import sys
import json
import datetime
import numpy as np
import skimage.draw
import cv2
from mrcnn.visualize import display_instances
import matplotlib.pyplot as plt

# Root directory of the project
ROOT_DIR = os.path.abspath("./")

# Import Mask RCNN
sys.path.append(ROOT_DIR)  # To find local version of the library
from mrcnn.config import Config
from mrcnn import model as modellib, utils

# Path to trained weights file
COCO_WEIGHTS_PATH = os.path.join(ROOT_DIR, "mask_rcnn_coco.h5")
\end{minted}

Definiamo la nostra configurazione di \mrcnn\ derivando la classe
\texttt{Config}.

\begin{minted}{python}
class CustomConfig(Config):
    """
    Configuration for training on the dataset provided on Google Drive.
    Derives from the base Config class and overrides some values.
    """
    # Give the configuration a recognizable name
    NAME = "document"

    # We use a GPU with 12GB memory, which can fit two images.
    # Adjust down if you use a smaller GPU.
    IMAGES_PER_GPU = 2

    # Number of classes (including background)
    NUM_CLASSES = 1 + 2  # Background + documenti

    # Number of training steps per epoch
    STEPS_PER_EPOCH = 150

    VALIDATION_STEPS = 25

    LEARNING_RATE=0.006

    # Skip detections with < 90% confidence
    DETECTION_MIN_CONFIDENCE = 0.9
\end{minted}

% TODO
CustomDataset, inherit the Dataset library class to override the
significant load mask method, necesssary to load the information about
the training image's data. the mask is the region of the image that
represent the objects that the net should recognize

%```
%############################################################
%#  Dataset
%############################################################
%
%class CustomDataset(utils.Dataset):
%    def load_custom(self, dataset_dir, is_train=True): # aggiungere modularita per diversi documenti
%        """
%          Load images from dataset directory, the images are stored in the structure own of the mask,
%          need to write good via_regions.json for every folder in the dataset
%
%          Parameters:
%          dataset_dir: path of the root folder of the dataset
%          is_train: boolean that represent the mode of the dataset to load
%        """
%
%        if is_train:
%          dataset_dir = os.path.join(dataset_dir, "training/")
%        else:
%          dataset_dir = os.path.join(dataset_dir, "validating/")
%
%        # Add classes.
%        self.add_class("dataset", 1, "tesserino")
%        self.add_class("dataset", 2, "patente")
%
%        # Load annotations
%        # VGG Image Annotator saves each image in the form:
%        # { 'filename': '28503151_5b5b7ec140_b.jpg',
%        #   'regions': {
%        #       '0': {
%        #           'region_attributes': {},
%        #           'shape_attributes': {
%        #               'all_points_x': [...],
%        #               'all_points_y': [...],
%        #               'name': 'polygon'}},
%        #       ... more regions ...
%        #   },
%        #   'size': 100202
%        # }
%        # We mostly care about the x and y coordinates of each region
%
%        dir_list = [dir_ for dir_ in os.listdir(dataset_dir) if os.path.isdir(os.path.join(dataset_dir, dir_))]
%
%        for dir_ in dir_list:
%          iter_dir = os.path.join(dataset_dir, dir_ + "/")
%
%          print("iterating: {0} ...".format(iter_dir))
%
%          annotations1 = json.load(open(os.path.join(iter_dir, "via_regions.json")))
%          annotations = list(annotations1.values())  # don't need the dict keys
%
%          # The VIA tool saves images in the JSON even if they don't have any
%          # annotations. Skip unannotated images.
%          annotations = [a for a in annotations if a['regions']]
%
%
%          # Add images
%          for a in annotations:
%              # Get the x, y coordinaets of points of the polygons that make up
%              # the outline of each object instance. There are stores in the
%              # shape_attributes (see json format above)
%              polygons = [(r['region_attributes'], r['shape_attributes']) for r in a['regions']]
%
%              # load_mask() needs the image size to convert polygons to masks.
%              # Unfortunately, VIA doesn't include it in JSON, so we must read
%              # the image. This is only managable since the dataset is tiny.
%              image_path = os.path.join(iter_dir, a['filename'])
%              image = skimage.io.imread(image_path)
%              height, width = image.shape[:2]
%
%              self.add_image(
%                  "dataset",
%                  image_id=a['filename'],  # use file name as a unique image id
%                  path=image_path,
%                  width=width, height=height,
%                  polygons=polygons)
%
%    # this method is invoked iterativelly from the net during training (GRADIENT DESCEND)
%    def load_mask(self, image_id):
%        """Generate instance masks for an image.
%       Returns:
%        masks: A bool array of shape [height, width, instance count] with
%            one mask per instance.
%        class_ids: a 1D array of class IDs of the instance masks.
%        """
%        # If not a bottle dataset image, delegate to parent class.
%        image_info = self.image_info[image_id]
%        if image_info["source"] != "dataset":
%            return super(self.__class__, self).load_mask(image_id)
%
%        # Convert polygons to a bitmap mask of shape
%        # [height, width, instance_count]
%        info = self.image_info[image_id]
%        mask = np.zeros([info["height"], info["width"], len(info["polygons"])],
%                        dtype=np.uint8)
%
%        class_ids = list()
%        for i, p in enumerate(info["polygons"]):
%            # Get indexes of pixels inside the polygon and set them to 1
%            rr, cc = skimage.draw.polygon(p[1]['all_points_y'], p[1]['all_points_x'])
%            mask[rr, cc, i] = 1
%            class_ids.append(self.class_names.index(p[0]['name']))
%
%        # Return mask, and array of class IDs of each instance. Since we have
%        # one class ID only, we return an array of 1s
%        return mask.astype(np.bool), np.asarray(class_ids, dtype='int32')
%
%    def image_reference(self, image_id):
%        """Return the path of the image."""
%        info = self.image_info[image_id]
%        print("reference...")
%        if info["source"] == "dataset":
%            print(info["path"])
%            return info["path"]
%        else:
%            super(self.__class__, self).image_reference(image_id)
%
%```
%
%
%```
%# Utilities
%
%classes_ = None
%
%def train(model):
%    """ Train the model. """
%
%    global classes_
%
%    # Training dataset.
%    dataset_train = CustomDataset()
%    dataset_train.load_custom("/content/drive/My Drive/Ing_sw_testing/document_dataset/", is_train = True)
%    dataset_train.prepare()
%
%    classes_ = dataset_train.class_names
%
%    # Validation dataset
%    dataset_val = CustomDataset()
%    dataset_val.load_custom("/content/drive/My Drive/Ing_sw_testing/document_dataset/", is_train = False) # cambiare directory
%    dataset_val.prepare()
%
%    # *** This training schedule is an example. Update to your needs ***
%    # Since we're using a very small dataset, and starting from
%    # COCO trained weights, we don't need to train too long. Also,
%    # no need to train all layers, just the heads should do it.
%    print("Training network heads")
%    model.train(dataset_train, dataset_val,
%                learning_rate=config.LEARNING_RATE,
%                epochs=10,
%                layers='heads')
%
%```
%
%initialize the data defined above to train our model
%
%
%```
%config = CustomConfig()
%config.display()
%```
%
%create the model in training mode and before train it we load the ms coco pretrained weights
%
%
%```
%# TRAINING
%model = modellib.MaskRCNN(mode="training", config=config,
%                                  model_dir="./")
%weights_path = COCO_WEIGHTS_PATH
%model.load_weights(weights_path, by_name=True, exclude=[
%            "mrcnn_class_logits", "mrcnn_bbox_fc",
%            "mrcnn_bbox", "mrcnn_mask"])
%
%train(model)
%
%```
%
%save the weights trained referred to my dataset
%
%
%```
%import time
%model_path = 'mask_rcnn_'  + '.' + str(time.time()) + '.h5'
%model.keras_model.save_weights(model_path)
%print(model_path)
%
%```
%
%download the weights calculated
%
%
%```
%from google.colab import files
%files.download(model_path)
%```
%
%define new ineheritanced config class, used only for detections
%
%
%```
%class InferenceConfig(CustomConfig):
%    # Set batch size to 1 since we'll be running inference on
%    # one image at a time. Batch size = GPU_COUNT * IMAGES_PER_GPU
%    GPU_COUNT = 1
%    IMAGES_PER_GPU = 1
%    DETECTION_MIN_CONFIDENCE = 0.9
%```
%
%
%```
%config = InferenceConfig()
%config.display()
%```
%
%create the model in inference mode for doing tests, loading the weights that we have just calculated
%
%
%```
%model = modellib.MaskRCNN(mode="inference", config=config,  model_dir="./")
%
%model.load_weights(model_path, by_name=True)
%```
%
%run detection on a testing never seen image
%
%
%```
%# Run object detection
%image = skimage.io.imread("/content/drive/My Drive/Ing_sw_testing/document_dataset/testing/test2.jpg")
%results = model.detect([image], verbose=1)
%
%# Display results
%r = results[0]
%
%print(classes_)
%
%display_instances(image, r['rois'], r['masks'], r['class_ids'],
%                            classes_, r['scores'],
%                            title="Predictions")
%
%print(r['rois'])
%
%```
%
%    Processing 1 images
%    image                    shape: (426, 640, 3)         min:    0.00000  max:  255.00000  uint8
%    molded_images            shape: (1, 1024, 1024, 3)    min: -123.70000  max:  138.20000  float64
%    image_metas              shape: (1, 15)               min:    0.00000  max: 1024.00000  float64
%    anchors                  shape: (1, 261888, 4)        min:   -0.35390  max:    1.29134  float32
%    ['BG', 'tesserino', 'patente']
%
%
%
%![png](document_recognition_29_1.png)
%
%
%    [[107 245 296 577]]


\end{document}
