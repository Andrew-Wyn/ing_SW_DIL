\documentclass[12pt,a4paper]{article}
\usepackage[T1]{fontenc}
\usepackage[utf8]{inputenc}
\usepackage[italian]{babel}
\usepackage{lmodern}
\usepackage{graphicx}
\usepackage{minted}

\setlength{\parindent}{2em}
\setlength{\parskip}{1em}

\begin{document}

\title{Riconoscimento Documenti}
\author{Simone Cimarelli \and Vittorio Mignini \and Luca Moroni}
\date{9 luglio 2020}

\maketitle

\begin{abstract}
    Si descrive l'implementazione di una libreria basata su Mask-RCNN e
    tesseract per il riconoscimento interattivo di documenti
    identificativi e tessere.
\end{abstract}

\section{Obbiettivi}

Implementare un applicativo che tramite l'utilizzo di un dispositivo di
input video sia in grado di riconoscere e di identificare un insieme di
documenti a priori definiti e di trascrivere in caratteri macchina delle
porzioni di testo presenti nel documento precedentemente identificato

\section{Analisi dei Requisiti}
\subsection{Analisi Requisiti Utente}

\begin{itemize}
    \item Ottenere un elenco dei documenti visibili all'interno di
        un'immagine e le componenti testuali specifiche in esso
        riconoscibili.
    \item Avere la possibilità di definire la natura dei documenti
        oggetto del riconoscimento e delle informazioni testuali
        rilevanti in essi contenute.
    \item Mettere a disposizione un interfaccia grafica semplice e
        minimale per l'acquisizione di video e immagini

\end{itemize}

\subsection{Analisi Requisiti Sistema}

\begin{itemize}
    \item Implementare delle tecnologie per l'object detection e ocr
        allo stato dell'arte per erogare in modo efficiente il servizio
        richiesto dal committente.
    \item Fornire strumenti che mettono a disposizione la possibilita di
        rendere scalabile il numero e la tipologia di documenti
        riconoscibili dall'applicativo
    \item L'applicativo finale \textit{dovrebbe} essere portatile e non
        strettamente dipendente da una particolare architettura
\end{itemize}

\subsection{Glossario e Dominio applicativo}

Il dominio applicativo di tale sistema è l'ambito della computer vision
e del ocr. In ultimo stadio l'utente finale avra a disposizione un
interfaccia grafica attraverso la quale in modo intuitivo ed efficiente
potra avere accesso al servizio erogato

\begin{description}
    \item[Computer Vision] Insieme di metodologie algoritmiche e
        matematiche che hanno come obiettivo quello di acquisire delle
        conoscenze ad alto livello partendo da immagini o video. Nel
        nostro caso capire se e dove nelle immagini passate
        all'applicativo è presente un documento del tipo prestabilio

    \item[Object Detection] tecnologia relativa all'ambito della
        computer vision, che propone una soluzione algoritmica per il
        riconoscimento di istanze riguardanti oggetti di classi
        predefinite in immagini digitali e video

    \item[OCR] Optical Character Recognition, metodologie software per
        il riconoscimento di caratteri contenuti in un documento e
        trascrivere tali simboli in caratteri equivalenti comprensibili
        dalla macchina.
\end{description}


\section{Architettura del Sistema}

\subsection{Scelte architetturali}

Abbiamo optato per lo sviluppo di una soluzione web, Il nostro
applicativo si suddivide in tre macro parti:

\begin{itemize}
    \item la parte frontend, sviluppata tramite tecnologie browser
        compatibili, js html CSS
    \item la parte backend, sviluppata in python tramite la libreria
        flask per la creazione di servizi web
    \item la parte di core bussness, ovvero il core dell'applicativo,
        sviluppata in python per l'implementazione di ocr e object
        detection
\end{itemize}

Nella progettazione architetturale di tale applicativo, si è optato per
l'utilizzo di librerie testate e rappresentanti lo stato dell'arte,
inerentemente all'ambito applicativo sopracitato.\\
Per quanto riguarda l'object detection, abbiamo selezionato la libreria
\textbf{Mask-R\_CNN} una libreria che oltre a object detection mette a
disposizione anche segmentazione di immagini, una feature aggiuntiva che
oltre a ritornare un rettangolo contenente un oggetto di una determinata
classe nell'immagine ne ritorna anche la regione di pixel che lo
rappresenta, tale feature comunque non è utilizzata nel nostro sistema,
ma è tornata utile in fase di training in quanto possiamo allenare la
nostra rete direttamente tramite i segmenti di documenti evitando la
presenza dei polpastrelli che in alcuni casi possono portare la nostra
rete fuori rotta.\\
Invece per effettuare l'ocr abbiamo optato per la libreria tesseract,
implementata in python tramite una libreria wrapper chiamata
pytesseract.

Le librerie appena citate hanno portato alla necessita di importarne
altre tra cui, tensorflow == 1.x, flask==2.0.8 sulle quali si basa
Mask-M\_CNN, opencv per la gestione e la codifica di immagini.

Per la gestione delle dipendenze in fase di sviluppo abbiamo utilizzato
due strumenti propri dell'ambiente python, pipenv che mette a
disposizione un ambiente virtuale contenente determinate librerie in
modo asettico rispetto a quelle di sistema e che rende ripetibile su
altre macchine la configurazione da noi utilizzata con semplici
passaggi. pyenv che invece mette a disposizione delle versioni di python
proprie in modo da non andare a modificare quello di sistema nel caso
sia necessaria un versione non corrispondente con quella di sistema, ad
esempio tensorflow1.x è utilizzabile solamente con versioni di python $
\leq $ 3.7.x con pyenv tale requisito puo essere soddisfatto su qualsiasi
macchina.

\subsection{Mask}

There are two stages of Mask RCNN. First, it generates proposals about
the regions where there might be an object based on the input image.
Second, it predicts the class of the object, refines the bounding box
and generates a mask in pixel level of the object based on the first
stage proposal. Both stages are connected to the backbone structure.

A light weight neural network called RPN scans all FPN top-bottom
pathway( hereinafter referred to feature map) and proposes regions which
may contain objects.

At the second stage, another neural network takes proposed regions by
the first stage and assign them to several specific areas of a feature
map level, scans these areas, and generates objects
classes(multi-categorical classified), bounding boxes and masks.

--> aggiungere descrizione piu dettagliata di mask-r\_cnn e foto descrittiva

\subsection{Training}

Essendo il nostro corebusiness basato su una rete neurale per la
computer vision implementata tramite la libreria Mask-R\_CNN, una rete
neurale molto complessa sulla quale facciamo transfer learning, ovvero
alleniamo solamente gli strati piu estrerni per adattare i pesi e la
struttura della stessa al riconoscimento di un certo numero di classi
definibili da noi sviluppatori e in ultima istanza dall'utente
utilizzatore, ci siamo appoggiati al servizio colab di google per il
training della rete tramite la computazione limitata ma estremamente
potente messa a dispozione da tale servizio.\\
Per rendere ripetibile tutto ciò abbiamo reso disponibile il dataset da
noi utilizzato e il file jupiter (tecnologia su cui si basa colab), cosi
da poter ripetere la fase di training facendo l'opportuno tuning sui
parametri e cosi da poter anche aggiungere nuove classi di oggetti e
nuove istanze per il training. Infine si rimanda al link
dell'applicativo web necessario alla creazione di annotazioni per le
immagini utilizzate come istanze di training.
\texttt{http://www.robots.ox.ac.uk/~vgg/software/via/}

--> aggiungere codice ipynb debitamente commentato

--> struttura di dataset con esempio di file di via\_regions.json

\subsection{Senza Nome}

Il parte di core business è implementata in un modulo Python. Le
funzionalità di riconoscimento sono rese disponibili tramite istanza di
una classe \texttt{Detectron}, inizializzata tramite una struttura di
configurazione definibile dall'utente in congiunzione con i pesi della
rete neurale, per la massima estensibilità.

Una volta inizializzato, un oggetto \texttt{Detectron} offre un metodo
\texttt{recognize} che implementa la funzione di riconoscimento.\\
\texttt{recognize} accetta un oggetto \texttt{bytes}-like che
rappresenta l'immagine in un qualsiasi formato supportato dalla libreria
OpenCV2. Dopo aver decodificato l'immagine questa viene passata
attraverso la rete neurale Mask-R\_CNN precedentemente inizializzata, che
individua delle ROI associate ai tipi di documento configurati, a ognuna
delle quali sarà associato un dizionario. L'area di ogni ROI viene
estratta dall'immagine e memorizzata alla chiave ``\texttt{snapshot}'',
codificata in PNG in un oggetto \texttt{bytes}.

Su questi snapshot viene poi applicato l'algoritmo di OCR della libreria
tesseract. Quest'ultimo a sua volta restituirà una serie di record che
rappresentano il testo trovato, il grado confidenza e la relativa
posizione all'interno della ROI. Questi record sono identificati in base
alla posizione o al formato, specificato tramite regex. Il testo dei
record così individuati è aggiunto a un dizionario sotto il nome di
``\texttt{attributes}''. L'oggetto di configurazione definisce quali di
questi ``\texttt{attributes}'' sono necessari al riconoscimento del
documento e solo quando tutti questi sono effettivamente stati
individuati il documento viene marcato come valido. Uno degli attributi
necessari può essere designato come chiave primaria, quando un documento
è valido il valore della chiave primaria viene restituito nel dizionario
sotto il nome di ``\texttt{primaryKey}'' e può essere utilizzato per
l'identificazione univoca del documento appena scansionato. Si consiglia
di indicare come chiave primaria il codice o numero identificativo
univoco del documento, quando presente.\\
La funzione \texttt{recognize} ritorna in output una lista di questi
dizionari.

\subsection{Backend}

--> descrizione backend

\subsection{Frontend}

--> descrizione frontend

\pagebreak

\subsection{Esempio di configurazione}
\begin{minted}{python}
{
    "weights": "percorso/ai/pesi.h5",
    "classes": {
        "Tesserino": {
            "lang": "ita",
            "min_conf": 50,
            "primaryKey": "Matricola",
            "regions": [
                {
                    "name": "Matricola",
                    "type": "regex",
                    "rule": "^\\d{6}$",
                    "needed": True
                },
                {
                    "name": "Nome",
                    "type": "box",
                    "rule": [0, 0.6, 0.2, 1],
                    "needed": False
                }
            ]
        }
    }
}
\end{minted}

\begin{figure}[p]
  \caption{Diagramma delle Classi}
  \centering
  \includegraphics[width=\textwidth,height=\textheight,keepaspectratio]{uml_class.pdf}
\end{figure}

--> aggiungere descrizione diagramma di classe

\begin{figure}[p]
  \caption{Diagramma di Stato}
  \centering
  \includegraphics[width=\textwidth,height=\textheight,keepaspectratio]{uml_state.pdf}
\end{figure}

--> aggiungere descrizione diagramma di stato per oggetto Detectron

\begin{figure}[p]
  \caption{Diagramma di Caso d'Uso}
  \centering
  \includegraphics[width=\textwidth,height=\textheight,keepaspectratio]{uml_use_case.pdf}
\end{figure}

--> aggiungere descrizione diagramma caso d'uso

\begin{figure}[p]
  \caption{Diagramma di Sequenza}
  \centering
  \includegraphics[width=\textwidth,height=\textheight,keepaspectratio]{uml_seq.pdf}
\end{figure}

--> aggiungere descrizione diagramma di sequenza (aggiungendo classe di libreria mrcnn)

\begin{figure}[p]
  \caption{Diagramma di Comunicazione}
  \centering
  \includegraphics[width=\textwidth,height=\textheight,keepaspectratio]{uml_comm.pdf}
\end{figure}

--> aggiungere descrizione diagramma di comunicazione (aggiungendo classe di libreria mrcnn)

\begin{figure}[p]
  \caption{Diagramma di Attività}
  \centering
  \includegraphics[width=\textwidth,height=\textheight,keepaspectratio]{uml_activity.pdf}
\end{figure}

--> aggiungere descrizione diagramma di attivita

\pagebreak

\section{Design Pattern}

Un design pattern che in potenza poterebbe essere applicato è il
Protection Proxy. Inizialmente non eravamo convinti della thread safness
della funzione \texttt{recognize} nella classe \texttt{Detectron}. Una
possibile soluzione consisterebbe nel creare una classe
\texttt{DetectronPool}, la quale nell'inizializzazione carica una pool
di oggetti \texttt{Detectron}. Tale classe avrà un metodo
\texttt{recognize} che rimanderà all'omonimo metodo di
\texttt{Detectron}, il quale gestirebbe l'allocazione della risorsa al
thread chiamante in modo trasparente tramite un semaforo contatore
inizializzato al numero di \texttt{Detectron} nella pool, garantendo la
mutua esclusione nell'esecuzione del metodo \texttt{recognize} in
maniera totalmente trasparente per l'utilizzatore.

\begin{figure}[h]
  \caption{Diagramma di Classe Proxy}
  \centering
  \includegraphics[width=\textwidth,height=\textheight,keepaspectratio]{uml_proxy.pdf}
\end{figure}

\end{document}
