\documentclass[12pt,A4]{article}
\usepackage[utf8]{inputenc}
\usepackage[italian]{babel}
\usepackage{tikz-uml}

\begin{document}

\title{Riconoscimento Documenti}
\author{Simone Cimarelli \and Vittorio Mignini \and Luca Moroni}
\date{9 luglio 2020}

\maketitle

\begin{abstract}
    %TODO
\end{abstract}

\section{Obbiettivi}

Implementare tramite una libreria Python l'object detection all'interno
di immagini di una serie di tipologie di documenti estensibili dal
client, e il riconoscimento dei caratteri per classi di dati specifiche
per gli stessi.\\
Sarà fornito anche una possibile implementazione di un frontend basato
su tecnologie browser

\section{Analisi dei Requisiti}
\subsection{Glossario}

\begin{description}
    \item[\textsc{CNN}] Convolutional Neural Network, rete neurale
        convoluzionale, adatta alla classificazione di intere immagini.

    \item[Mask \textsc{R-CNN}] egion-based Convolutional Neural Network,
        libreria basata su Resnet 101 e tecnologia \textsc{CNN} in grado
        di effettuare segmentazione di immagini.

    \item[Segmentazione] Individuazione delle aree di un'immagine
        contenenti determinati oggetti.

    \item[Training] ``Allenamento'' della rete neurale al riconoscimento
        di uno o più tipi di documento

    \item[\textsc{OCR}] Optical Character Recognition, riconoscimento
        ottico del testo di un documento.

\end{description}

\subsection{Analisi Requisiti Utente}

\begin{itemize}
    \item Ottenere un elenco dei documenti visibili all'interno di
        un'immagine e le componenti testuali specifiche in esso
        riconoscibili.
    \item Avere la possibilità di definire la natura dei documenti
        oggetto del riconoscimento e delle informazioni testuali
        rilevanti in essi contenute
    \item Fornire strumenti per il retraining della rete neurale per
        nuovi tipi di documento
\end{itemize}

\subsection{Analisi Requisiti Sistema}

\begin{itemize}
    \item Target: Python 3.7, Tensorflow 1.13.1, Keras 2.0.8, ultime
        versioni certamente compatibili con la libreria Mask
        \textsc{R-CNN} utilizzata in questo progetto
    \item Mettere a disposizione un'implementazione di riferimento di
        frontend compatibile con la libreria
\end{itemize}

\section{Architettura del Sistema}

\begin{tikzpicture}
    \begin{umlpackage}[x=0, y=0]{mrcnn}
        \umlclass[x=0, y=0]{MRCNN} {
            + keras\_model
        }{
            + load\_weights(file\_path, by\_name)\\
            + detect(images, verbose)
        }
        \umlsimpleclass[x=5, y=1]{Config}
        \umlclass[]{Dataset}{
        }{
            + load\_mask\\
            + image\_reference
        }
    \end{umlpackage}

    \begin{umlpackage}[x=0, y=-6]{training}
        \umlclass[x=0, y=0]{TrainingConfig} {
            + NAME\\
            + IMAGES\_PER\_GPU\\
            + NUM\_CLASSES\\
            + STEPS\_PER\_EPOCH\\
            + LEARNING\_RATE\\
            + DETECTION\_MIN\_CONFIDENCE
        }{}
        \umlclass[]{CustomDataset}{
        }{
            + load\_custom\\
            + load\_mask\\
            + image\_reference
        }
    \end{umlpackage}

    \begin{umlpackage}[x=0, y=-12]{documentCNN}
        \umlclass[x=0, y=0]{OcrRecord}{
            + conf\\
            + text\\
            + x\\
            + y\\
            + w\\
            + h
        }{}
        \umlclass[x=0, y=0]{InferenceConfig} {
            + NAME\\
            + IMAGES\_PER\_GPU\\
            + NUM\_CLASSES\\
            + STEPS\_PER\_EPOCH\\
            + LEARNING\_RATE\\
            + DETECTION\_MIN\_CONFIDENCE
        }{}
        \umlclass[]{Detectron}{
            - model\\
            - classes\\
            - class\_names
        }{
            + recognize(image) : dict
        }
    \end{umlpackage}
\end{tikzpicture}

\end{document}
