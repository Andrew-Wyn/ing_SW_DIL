\documentclass[12pt,a4paper]{article}
\usepackage[T1]{fontenc}
\usepackage[utf8]{inputenc}
\usepackage[italian]{babel}
\usepackage{lmodern}
\usepackage{graphicx}

\begin{document}

\title{Riconoscimento Documenti}
\author{Simone Cimarelli \and Vittorio Mignini \and Luca Moroni}
\date{9 luglio 2020}

\maketitle

\begin{abstract}
    %TODO
\end{abstract}

\section{Obbiettivi}

Implementare tramite una libreria Python l'object detection all'interno
di immagini di una serie di tipologie di documenti estensibili dal
client, e il riconoscimento dei caratteri per classi di dati specifiche
per gli stessi.\\
Sarà fornito anche una possibile implementazione di un frontend basato
su tecnologie browser

\section{Analisi dei Requisiti}
\subsection{Glossario}

\begin{description}
    \item[CNN] Convolutional Neural Network, rete neurale
        convoluzionale, adatta alla classificazione di intere immagini.

    \item[Mask R-CNN] egion-based Convolutional Neural Network, libreria
        basata su Resnet 101 e tecnologia CNN in grado di effettuare
        segmentazione di immagini.

    \item[Segmentazione] Individuazione delle aree di un'immagine
        contenenti determinati oggetti.

    \item[Training] ``Allenamento'' della rete neurale al riconoscimento
        di uno o più tipi di documento

    \item[OCR] Optical Character Recognition, riconoscimento ottico del
        testo di un documento.

\end{description}

\subsection{Analisi Requisiti Utente}

\begin{itemize}
    \item Ottenere un elenco dei documenti visibili all'interno di
        un'immagine e le componenti testuali specifiche in esso
        riconoscibili.
    \item Avere la possibilità di definire la natura dei documenti
        oggetto del riconoscimento e delle informazioni testuali
        rilevanti in essi contenute
    \item Fornire strumenti per il retraining della rete neurale per
        nuovi tipi di documento
\end{itemize}

\subsection{Analisi Requisiti Sistema}

\begin{itemize}
    \item Target: Python 3.7, Tensorflow 1.13.1, Keras 2.0.8, ultime
        versioni certamente compatibili con la libreria Mask R-CNN
        utilizzata in questo progetto
    \item Mettere a disposizione un'implementazione di riferimento di
        frontend compatibile con la libreria
\end{itemize}

\section{Architettura del Sistema}

Lorem ipsum dolor sit amet, consectetur adipisci elit, sed do eiusmod
tempor incidunt ut labore et dolore magna aliqua. Ut enim ad minim
veniam, quis nostrum exercitationem ullamco laboriosam, nisi ut aliquid
ex ea commodi consequatur. Duis aute irure reprehenderit in voluptate
velit esse cillum dolore eu fugiat nulla pariatur. Excepteur sint
obcaecat cupiditat non proident, sunt in culpa qui officia deserunt
mollit anim id est laborum.

\begin{figure}[p]
  \caption{Diagramma delle Classi}
  \centering
  \includegraphics[width=1\textwidth]{uml_class.pdf}
\end{figure}

\begin{figure}[p]
  \caption{Diagramma di Sequenza}
  \centering
  \includegraphics[width=1\textwidth]{uml_seq.pdf}
\end{figure}

\begin{figure}[p]
  \caption{Diagramma di Comunicazione}
  \centering
  \includegraphics[width=1\textwidth]{uml_comm.pdf}
\end{figure}

\end{document}
